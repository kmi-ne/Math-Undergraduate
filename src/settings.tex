%%%%% ページレイアウト
% \setpagelayout+{vmargin=10truemm, inner=15truemm, outer=10truemm, includemp, marginparwidth=35truemm}
\setpagelayout+{vmargin=8truemm, hmargin=10truemm, includemp, nomarginpar}
% \usepackage{marginnote}
% \renewcommand*{\marginfont}{\scriptsize}

% 
% 
% 

%%%%% フォント/書体
\ifxetex
  \setCJKsansfont[BoldFont=HaranoAjiGothic-Bold]{HaranoAjiGothic-Medium}
  \setCJKmonofont[BoldFont=HaranoAjiGothic-Bold]{HaranoAjiGothic-Medium}
\else\fi
\ifluatex
  \ltjsetparameter{jacharrange={-2,-3}}
  \setsansjfont[BoldFont=HaranoAjiGothic-Bold]{HaranoAjiGothic-Medium}
  \setmonojfont[BoldFont=HaranoAjiGothic-Bold]{HaranoAjiGothic-Medium}
\else\fi
\PassOptionsToPackage{math-style=ISO, bold-style=upright}{unicode-math}
\usepackage[olddefault, newcmbb, varnothing]{fontsetup}
\setmathfont[range={scr, bfscr}, StylisticSet=1]{NewCMMath-Regular}
\DeclareMathAlphabet{\cl}{OMS}{cmsy}{m}{n} % 旧チャンセリー筆記体
\setsansfont{nimbussans}
\usepackage{manfnt}
\usepackage[old]{old-arrows}
\ifluatex
  \usepackage{longmath}
\else\fi
\makeatletter
\AtBeginDocument{
  \catcode`_=12
  \begingroup\lccode`~=`_
  \lowercase{\endgroup\let~}\sb
  \mathcode`_="8000
  \immediate\write\@auxout{\catcode`_=12 }
}
\makeatother
\allowdisplaybreaks[4]

% 
% 
% 

%%%%% 色
\usepackage{xcolor}
\definecolor{mygreen}{HTML}{70b548}
\definecolor{myblue}{HTML}{6e9ec2}
\definecolor{myred}{HTML}{c26e7c}
\definecolor{myyellow}{HTML}{b8ba05}
\definecolor{mygray}{HTML}{878787}
\definecolor{myurlcolor}{HTML}{ad0d00}
\definecolor{mylinkcolor}{HTML}{2a4f99}
\newcommand{\gx}[1]{\textcolor{mygreen}{#1}}
\newcommand{\bx}[1]{\textcolor{myblue}{#1}}
\newcommand{\rx}[1]{\textcolor{myred}{#1}}
\newcommand{\Gx}[1]{\textcolor{mygray}{#1}}

% 
% 
% 

%%%%% 目次
\setcounter{tocdepth}{4}

% 
% 
% 

%%%%% ヘッダー/フッター/見出し
\renewcommand{\plainifnotempty}{} % クラスファイルによるページスタイル強制変更を除去
\usepackage[pagestyles, clearempty, toctitles, explicit]{titlesec}
\newcommand{\headsize}{\small}

\makeatletter
\newcommand{\labelchapter}{\@chapapp \thechapter \@chappos}
\newcommand{\labelsection}{\@secapp \thesection \@secpos}
\makeatother

%%% ヘッダー/フッター
\newpagestyle{nc-ns}[\headsize\gs]{%
  \headrule
  \sethead[\textit{\thepage}][\labelchapter\quad\chaptertitle][]% 偶数ページ
  {}{\labelsection\quad\sectiontitle}{\textit{\thepage}}% 奇数ページ
}
\newpagestyle{nc-s}[\headsize\gs]{%
  \headrule
  \sethead[\textit{\thepage}][\labelchapter\quad\chaptertitle][]% 偶数ページ
  {}{\sectiontitle}{\textit{\thepage}}% 奇数ページ
}
\newpagestyle{nc-nc}[\headsize\gs]{%
  \headrule
  \sethead[\textit{\thepage}][\labelchapter\quad\chaptertitle][]% 偶数ページ
  {}{\labelchapter\quad\chaptertitle}{\textit{\thepage}}% 奇数ページ
}
\newpagestyle{c-c}[\headsize\gs]{%
  \headrule
  \sethead[\textit{\thepage}][\chaptertitle][]% 偶数ページ
  {}{\chaptertitle}{\textit{\thepage}}% 奇数ページ
}

%%% 見出し
\titleformat{\chapter}[display]{\thispagestyle{empty}\filleft\gsb}{\fontsize{80}{0}\selectfont\thechapter}{20pt}{\huge #1}
\titlespacing*{\chapter}{0pt}{50pt}{40pt}

\titleformat{\section}[display]{\gs}{\large\labelsection\hspace{.5em}\titlerule}{3pt}{\Large #1}
\titleformat{name=\section,numberless}{\gs\Large}{}{0em}{#1\hspace{.5em}\titlerule}
\titlespacing*{\section}{0pt}{15pt}{10pt}
\titlespacing*{name=\section,numberless}{0pt}{15pt}{10pt}

\titleformat{\subsection}{\gs\large}{\thesubsection}{1em}{#1}

\renewcommand{\jsParagraphMark}{}

\usepackage{froufrou}

%%% 章末問題
\newcommand{\exercises}{
  \pagestyle{nc-s}
  \sectionmark{章末問題}
  \phantomsection
  \section*{章末問題}
  \addcontentsline{toc}{section}{章末問題}
}

% 
% 
% 

%%%%% ハイパーリンク
\usepackage{hyperref}
\hypersetup{setpagesize=false, bookmarksdepth=4, bookmarksnumbered=true, colorlinks=true, linkcolor=mylinkcolor, citecolor=mylinkcolor, urlcolor=myurlcolor, pdfauthor={鴎海}}
\newcommand{\phantomlabel}[1]{\phantomsection\label{#1}}
\newcommand{\phyperref}[2]{\hyperref[#1]{#2\textsuperscript{→p.\,\pageref*{#1}}}}

% 
% 
% 

%%%%% cref
\usepackage[nameinlink]{cleveref}
\crefname{page}{p.}{pp.}
\makeatletter
\crefformat{chapter}{#2\@chapapp #1\@chappos #3}
\makeatother
\crefformat{section}{#2\S #1#3}
\crefformat{subsection}{#2\S\S #1#3}
\crefname{equation}{式}{式}
\creflabelformat{equation}{#2(#1)#3}
% \crefname{figure}{図}{図}
% \crefname{table}{表}{表}

% 
% 
% 

%%%%% zref
\usepackage{zref-xr}
\zxrsetup{toltxlabel}

% 
% 
% 

%%%%% リスト
\usepackage{tasks}
\usepackage[inline]{enumitem}
\setlist{align=left, leftmargin=*}
% - - - - - - - - - -
\newlist{myenum}{enumerate}{3}
\setlist[myenum, 1]{label=(\roman{myenumi}), ref=(\roman{myenumi})}
\setlist[myenum, 2]{label=(\roman{myenumi}.\roman{myenumii}), ref=(\roman{myenumi}.\roman{myenumii})}
\setlist[myenum, 3]{label=(\roman{myenumi}.\roman{myenumii}.\roman{myenumiii}), ref=(\roman{myenumi}.\roman{myenumii}.\roman{myenumiii})}
\newlist{myenum*}{enumerate*}{1}
\setlist[myenum*]{label=(\roman*), ref=(\roman*)}
% - - - - - - - - - -
\newlist{thmlist}{enumerate}{3}
\setlist[thmlist, 1]{label=\arabic{thmlisti}., ref=\arabic{thmlisti}}
\setlist[thmlist, 2]{label=\arabic{thmlisti}.\arabic{thmlistii}., ref=\arabic{thmlisti}.\arabic{thmlistii}.}
\setlist[thmlist, 3]{label=\arabic{thmlisti}.\arabic{thmlistii}.\arabic{thmlistiii}., ref=\arabic{thmlisti}.\arabic{thmlistii}.\arabic{thmlistiii}.}
\makeatletter
\renewcommand{\p@thmlisti}{\perh@ps{\thetcbcounter.}}
\protected\def\perh@ps#1{#1}
\newcommand{\itemref}[1]{\begingroup\let\perh@ps\@gobble\ref{#1}\endgroup}
\makeatother
% - - - - - - - - - -
\newlist{step}{enumerate}{3}
\setlist[step, 1]{label=(\arabic{stepi}), ref=(\arabic{stepi})}
\setlist[step, 2]{label=(\arabic{stepi}.\arabic{stepii}), ref=(\arabic{stepi}.\arabic{stepii})}
\setlist[step, 3]{label=(\arabic{stepi}.\arabic{stepii}.\arabic{stepiii}), ref=(\arabic{stepi}.\arabic{stepii}.\arabic{stepiii})}
% - - - - - - - - - -
\newlist{exc}{enumerate}{1}
\setlist[exc]{label=問\arabic*., ref=\arabic*}
\crefname{exci}{問}{問}
% - - - - - - - - - -
% descriptionのitemへのリンク
\makeatletter
\NewDocumentCommand\itemlab{mmO{}}{
  \item[#1]
  \phantomsection
  \ifstrempty{#3}{\renewcommand{\@currentlabel}{#1}}{\renewcommand{\@currentlabel}{#3}}
  \label{#2}
}
\makeatother

\usepackage{tabto}

% 
% 
% 

%%%%% tcolorbox
\usepackage{tcolorbox}
\tcbuselibrary{skins, breakable}
\tcbsetforeverylayer{myitm/.style={%
  enhanced,
  breakable,
  colbacklower=white, coltitle=white, skin=bicolor,
  fonttitle=\gsb,
  boxrule=1pt,
  sharp corners,
}}
\NewTColorBox[auto counter, number within=section]{none}{}{}
\NewDocumentCommand{\defitem}{mmmm}{%
  \crefname{thmlist#1}{#2}{#2}
  \NewTColorBox[use counter from=none, crefname={#2}{#2}]{#1}{O{}O{}O{}}{%
    myitm,
    colframe=#3, colback=#3!10!white,
    title={#2\thetcbcounter\ifstrempty{##3}{}{(##3)}},
    label={\ifstrempty{##1}{item_\thetcbcounter}{##1}},
    nameref={\ifstrempty{##2}{##3}{##2}},
    before upper={\crefalias{thmlisti}{thmlist#1}},
    after upper={\labellist{##1}},
    before lower={{\gsb #4:}\hspace{1em}},%
  }%
}
\newcommand{\tcbex}{{\gsb 例:}\hspace{1em}}
\defitem{rgl}{規約}{myblue}{例}
\defitem{dfn}{定義}{myblue}{例}
\defitem{mdfn}{メタ定義}{myblue}{例}
\defitem{lem}{補題}{mygreen}{証明}
\defitem{prp}{命題}{mygreen}{証明}
\defitem{mthm}{メタ定理}{mygreen}{証明}
\defitem{thm}{定理}{mygreen}{証明}
\defitem{asm}{前提}{myred}{例}
\defitem{axm}{公理}{myred}{例}
\defitem{cnv}{約束}{mygray}{例}

% 注意
\NewTColorBox{nb}{O{}}{%
  enhanced,
  breakable,
  colframe=black, colback=white, coltitle=black, colbacktitle=white,
  borderline={.7pt}{0pt}{dashed},
  frame hidden,
  attach boxed title to top text left={yshift*=-\tcboxedtitleheight/2},
  boxed title style={tile}, boxed title size=title,
  title={\gs 注意\ifstrempty{#1}{}{(#1)}},
}

\makeatletter
\newcommand{\setlabelname}[1]{\protected@edef\@currentlabelname{#1}}
\makeatother
\ExplSyntaxOn
\bool_if:NTF \g_my_UseLabelList_bool {
  \seq_new:N \g_my_labellist_seq
  \NewDocumentCommand { \mylabel } { mO{} } {
    \tl_if_blank:nTF { #2 } {} { \setlabelname{#2} }
    \label{#1}
    \seq_gput_right:Nn \g_my_labellist_seq { #1 }
  }
  \NewDocumentCommand { \labellist } { m } {
    \group_begin:
    \footnotesize
    \tl_if_blank:nTF { #1 } {} {
      \hfill ---\ label:\ \texttt{#1}
    }
    \seq_if_empty:NTF \g_my_labellist_seq {} {
      \tl_if_blank:nTF { #1 } {} { \\ }
      \hfill ---\ label:\
      \seq_set_map:NNn \l_tmpa_seq \g_my_labellist_seq { \texttt{##1} }
      \seq_use:Nn \l_tmpa_seq { \\ \hfill }
    }
    \seq_gclear:N \g_my_labellist_seq
    \group_end:
  }
} {
  \NewDocumentCommand { \mylabel } { mO{} } { \label{#1} }
  \NewDocumentCommand { \labellist } { m } {}
}
\ExplSyntaxOff


% 
% 
% 

%%%%% 表など
\usepackage{xltabular}
\usepackage{booktabs}

% 
% 
% 

%%%%% TikZ
% \usetikzlibrary{}

% 
% 
% 

%%%%% 索引
\ExplSyntaxOn
\keys_define:nn { idx } {
  parent .tl_set:N = \l_idx_parent_tl,
  parent-sort .tl_set:N = \l_idx_parentsort_tl,
  real .tl_set:N = \l_idx_real_tl
}
\NewDocumentCommand { \idx } { mmO{}O{} } {
  \group_begin:
  \keys_set:nn { idx } { #4 }
  \tl_if_empty:NTF \l_idx_parent_tl {
    \index{#2@#1}
  } {
    \index{\l_idx_parentsort_tl@\l_idx_parent_tl!#2@#1}
  }
  \tl_if_empty:NTF \l_idx_real_tl {
    \textsf{#1}
  } {
    \textsf{\l_idx_real_tl}
  }
  \tl_if_empty:nTF { #3 } {} { (#3) }
  \group_end:
}

\bool_if:NTF \g_my_UseIndex_bool {
  \usepackage{makeidx}
  \makeindex
} {
  \newcommand{\printindex}{}
  \RenewDocumentCommand{\idx}{mmO{}O{}}{\textsf{#1}\ifstrempty{#3}{}{(#3)}}
}
\ExplSyntaxOff

% 
% 
% 

%%%%% 参考文献
\ExplSyntaxOn
\bool_if:NTF \g_my_UseBib_bool {
  \usepackage[style=authoryear, backend=biber]{biblatex}
  \addbibresource{biblio.bib}
  \DeclareNameAlias{default}{family-given}
  \newcommand{\myprintbibliography}{
    \nocite{*}
    \printbibliography[title=参考文献, label=biblio]
  }
} {
  \newcommand{\myprintbibliography}{}
}
\ExplSyntaxOff

% 
% 
% 

%%%%% グロッサリー
\ExplSyntaxOn
\bool_if:NTF \g_my_UseGls_bool {
  \usepackage[record]{glossaries-extra}
  \GlsXtrLoadResources[src=symbols, set-widest, sort=use]
  \newcommand{\myprintglossary}{\printunsrtglossary[style=symbol-list]}
  \newglossarystyle{symbol-list}{
    \renewenvironment{theglossary}{
      \begin{longtable}[l]{@{}cll@{}}
    }{%
      \end{longtable}
    }
    \renewcommand*{\glossaryheader}{%
      \toprule
      {\gs 記号} & {\gs 説明} & {\gs 項目} \\
      \midrule
      \endfirsthead
      \toprule
      {\gs 記号} & {\gs 説明} & {\gs 項目} \\
      \midrule
      \endhead
      \bottomrule
      \endfoot
      \bottomrule
      \endlastfoot
    }
    \renewcommand*{\glossentry}[2]{%
      \glossentrysymbol{##1} & \glossentrydesc{##1} & \glstarget{##1}{\glossentryname{##1}} \tabularnewline
    }
    \renewcommand*{\glossarytitle}{記号一覧}
    \renewcommand*{\glossarytoctitle}{記号一覧}
  }
} {
  \newcommand{\glsadd}[1]{}
  \newcommand{\myprintglossary}{}
}
\ExplSyntaxOff

% 
% 
% 

%%%%% 数式
\everymath=\expandafter{\the\everymath \narrowbaselines\displaystyle}
