\section{メタ言語と対象言語}



\begin{rgl}[dfn_shape]
  \idx{図形}{ずけい}[shape]とは,紙などの媒体上に描画される,視覚的に認識可能な形状をいう.以下,任意の図形を,大文字または小文字ラテン文字の筆記体($𝒜, \… , 𝒵, 𝒶, \… , 𝓏$)で表す.
\end{rgl}



\begin{rgl}
  \begin{thmlist}
    \item \mylabel{dfn_sameShape}
    図形$𝒳$と図形$𝒴$が\idx{同じである}{おなじである}[same][parent={図形が}, parent-sort={ずけいが}]とは,$𝒳$と$𝒴$が同じ形状であることをいい,$𝒳 ≡ 𝒴$と書かれる.
    \item \mylabel{dfn_differentShape}
    図形$𝒳$と図形$𝒴$が\idx{異なる}{ことなる}[different][parent={図形が}, parent-sort={ずけいが}]とは,$𝒳$と$𝒴$が同じでないことをいい,$𝒳 ≢ 𝒴$と書かれる.
  \end{thmlist}
\end{rgl}



\begin{rgl}[dfn_language]
  \idx{言語}{げんご}[language]$L$は,以下によって特徴づけられる.
  \begin{thmlist}
    \item \mylabel{dfn_alphabet}
    言語$L$に応じて図形がいくつか選ばれ,それらは$L$の\idx{アルファベット}{あるふぁべっと}[alphabet]と呼ばれる.なお,どの図形を$L$のアルファベットとして選ぶかは$L$に依存し,その選び方は明示される場合もされない場合もある.
  \end{thmlist}
  以下,任意の言語を$L$や$M$などで表す.
\end{rgl}



\begin{rgl}[dfn_expression]
  図形$𝒳$が言語$L$に属する\idx{表現}{ひょうげん}[expression]であるということを,以下の条件によって定める.
  \begin{myenum}
    \item $L$のアルファベットは$L$に属する表現である.
    \item 図形$𝒴$が$L$に属する表現であり,図形$𝒶$が$L$のアルファベットであれば,$𝒴$の右に$𝒶$を描画してできる図形$𝒴𝒶$は$L$に属する表現である.
    \item 図形$𝒳$が$L$に属する表現であれば,以下のいずれかが成り立つ.
    \begin{myenum}
      \item $𝒳$は$L$のアルファベットである.
      \item $𝒳 ≡ 𝒴𝒶$であるような,$L$に属する表現$𝒴$と,$L$のアルファベット$𝒶$が存在する.
    \end{myenum}
  \end{myenum}
  $𝒳$が$L$に属する表現であるということを,$𝒳 : \Expr L$と書く.
\end{rgl}



\begin{rgl}
  \begin{thmlist}
    \item \mylabel{dfn_correspondence}
    単に\idx{対応}{たいおう}[correspondence]というとき,それはある図形に対して,ある図形を一つだけ対応させることをいう.図形$𝒳$を図形$𝒴$に対応させる対応を,$𝒳 ⥛ 𝒴$と書く.
    \item \mylabel{dfn_assignment}
    \idx{割り当て}{わりあて}[assignment]とは,いくつかの対応の集まりをいう.
  \end{thmlist}
\end{rgl}



\begin{rgl}[dfn_naming]
  $L$,$M$を言語とする.$M$による$L$への\idx{名付け}{なづけ}[naming]とは,以下を満たす対応$𝒳 ⥛ 𝒴$全体からなる対応をいう.
  \begin{myenum}
    \item $𝒳 : \Expr L$かつ$𝒴 : \Expr M$
  \end{myenum}
\end{rgl}
